\documentclass{article}

%add packages here
\usepackage{amsmath}
\usepackage[parfill]{parskip}
\usepackage{enumitem}
\usepackage{amssymb}
%no packages after this point

%
% Basic Document Settings
%
\topmargin=-0.45in
\evensidemargin=0in
\oddsidemargin=0in
\textwidth=6.5in
\textheight=9.0in
\headsep=0.25in

%title 
\title{HW1}
\date{\today}
\author{Pedram Safaei}

\begin{document}
	%title page and make sure it doesn't number the title page
	\pagenumbering{gobble}
	\maketitle
	\newpage
	\pagenumbering{arabic}
	%title page over
	
	%section{} will number the section
	%section*{} will not number the section
	\section*{Problem 1} 
	
	1. Find the probability that a random number from 1 to 1000 is divisible by 2 or 3.
	
	
	\begin{align*} 
	A &= \{Divisible by 2\}  \rightarrow \frac {1000}{2} \rightarrow 500 \quad (\text{Integer part})\\
	B &= \{Divisible by 3\}  \rightarrow \frac {1000}{3} \rightarrow 333 \quad (\text{Integer part})\\
	A \cap B &= \{Divisible by 6\}  \rightarrow \frac {1000}{6} \rightarrow 166  \quad (\text{Integer part})\\
	P(A) &= \frac{500}{100} = \frac{1}{2}\\
	P(B) &= \frac{333}{1000}\\
	P(A \cap B) &= \frac{166}{1000}\\
	P(A \cup B) &= \frac{500}{1000} + \frac{333}{1000} - \frac{166}{1000} = \boldsymbol{\frac{667}{1000}}
	\end{align*}
	
	
	\section*{Problem 2} 
	2. For events A and B such that 
	\begin{equation*}
	P ( A ) = 0.5 , P ( B ) = 0.4 , P ( A \cap B ) = 0.2 
	\end{equation*}
	find the probability that either A or B happened but not both.
	
	\begin{align*}
	P(A \cup B) &= P(A) + P(B)  - P(A \cap B)\\
	&= 0.5 + 0.4 - .2 = \boldsymbol{0.7}
	\end{align*}
	
	\section*{Problem 3} 
	3. (SOA) Upon arrival at a hospital emergency room, patients are categorized according to their condition as critical, serious, or stable. In the past year, 10\% of the ER patients were critical, 30\% serious, and the rest stable. Moreover, 40\% of critical patients died, 10\% of serious patients died, and 1\% of stable patients died. Given that the patient survived, what is the probability that this patient was categorized as serious?
	
	\begin{align*}
	a &= \text{patient is critical} \\
	b &= \text{patient is serious} \\
	c &= \text{patient is stable} \\
	S &= \text{patient survived }\\
	P(a) &= 0.1 \\
	P(b) &= 0.3\\
	P(c) &= 0.6\\
	P(S \mid a ) &= 0.6\\
	P(S \mid b) &= 0.9\\
	P(S \mid c) &= 0.99\\
	% P(\text{patient was categorized as serious given that patient survived}) &= P(b \mid S) \\
	P(b \mid S) & = \frac{P(b) P(S \mid b)}{P(S \mid a) P(a) + P(S \mid b) P(b)+ P(S \mid c) P(c)}\\
	&= \frac{.9 * .3}{.99 * .6 + .9 * .3 + .6 * .1}\\
	&= .29 = \boldsymbol{29\%} \\
	\end{align*}
	
	\section*{Problem 4} 
	4. (SOA) For two events A and B, determine the probability of A if 
	\begin{equation*}
     P(A\cup B) = 0.7,\ P(A\cup B^c) = 0.9.
	\end{equation*}
	
	\begin{align*}
	P(A\cup B) &= P(A) + P(B) - P(A\cap B)\\
	P(A\cup B') &= P(A) + P(B') - P(A\cap B')\\
	P(A\cup B) + P(A\cup B') &= P(A) + P(B) - P(A\cap B) + P(A) + P(B') - P(A\cap B')\\
	P(A\cap B) + P(A\cap B') &= P(A) \& P(B) + P(B') = 1\\
	P(A\cup B) + P(A\cup B') &= P(A) + 1\\
	P(A) &= P(A\cup B) + P(A\cup B') - 1\\
	&= 0.7 + 0.9 - 1 = \boldsymbol{0.6}\\
	\end{align*}
	
	\section*{Problem 5} 
	5. Assume that 0.1\% of people from a certain population have a germ. A test gives false positive (that is, shows a person has this germ if this person does not actually have a germ) in 10\% of cases when the person does not have this germ. This test gives false negative (shows a person does not have this germ if this person actually has it) in 20\% of cases when this person has this germ. Suppose you pick a random person from the population and apply this test twice. Both times it gives you positive result (that is, the test says that this person has this germ). What is the probability that this person actually has this germ?
	
		\begin{align*}
	a &= \text{Person has germ} \\
	b &= \text{Person doesn't have germ} \\
	S &= \text{person Shows positive both times}\\
	P(a) &= 0.001 \\
	P(b) &= 0.999\\
	P(S \mid a ) &= (1-.2)^2 = .64\\
	P(S \mid b) &= .1^2 = .01\\
	P(a \mid S) & = \frac{P(a) P(S \mid a)}{P(S \mid a) P(a) + P(S \mid b) P(b)}\\
	&= \frac{.001 * .64}{.001 * .64 + .999 * .01 }\\
	&= .0602 = \boldsymbol{6\%} \\
	\end{align*}
	
	\section*{Problem 6} 
6. A dashboard warning light is supposed to flash red if the car's pressure is too low. The probability of light flashing when it should is 98\%, and 3\% of the time it flashes for no reason. If there is a 10\% chance that the oil pressure is low, what is the chance that the driver should be concerned when light is flashing?

\begin{align*}
P(\text{Need for Concern \& ligh Flashes}) &= 0.1 * 0.98 = 0.098  \rightarrow (1) \\
P(\text{No Need for Concern \& ligh Flashes}) &= 0.9 * 0.03 = 0.027 \rightarrow (2) \\
P(\text{Light Flashes}) &= (1) + (2) \\
&= 0.125 \rightarrow (3) \\
P(\text{Need for concern} \mid \text{Light Flahses}) &= \frac{(1)}{(3)}\\
&= \frac{.098}{.125}\\
&= \boldsymbol{0.784}
\end{align*}


	\section*{Problem 7} 
	7. A real estate agent has 20 clients from Reno, 14 clients from Sparks, and 6 clients from Carson City. The agent will choose today's client at random, uniformly. Each communication with a client results in a closed deal with probability 20\% if this client is from Reno, 25\% from Sparks, and 15\% from Carson City. Find the probability that today the deal will be closed. 
	
	\begin{align*}
	P(\text{Today the deal Closes}) = & P(\text{Client From Reno \& Deal Closes}) + \\
	& P(\text{Client From Sparks \& Deal Closes}) + \\
	& P(\text{Client From Carson City \& Deal Closes}) \\
	= & (\frac{20}{40}) * 0.2 + \\
	& (\frac{14}{40}) * 0.25 + \\
	& (\frac{6}{40}) * 0.15 \\
	=& \boldsymbol{0.21}
	\end{align*}

	\section*{Problem 8} 
	8. The probability of A is 0.7, the probability of B is 0.8. What are the possible values for the probability of both A and B happening?
	
	\begin{align*}
	P(A) &= 0.7 \\
	P(B) &= 0.8 \\
    P(A \cap B) &= P(A) + P(B) - P(A \cup B) \\
	P(A \cap B) &= 0.7 + 0.8 - P(A \cup B)\\
    P(A \cap B) &= 1.5 - P(A \cup B)\\
	 \text{Maximum value for }P(A \cup B) \text{\quad is} \quad1\\
	 \text{Minumum value for } P(A \cup B) \text{\quad is} \quad MAX(P(A), P(B))  \\
	\text{Because $A \cup B$ is union of A \& B and $P(A \cup B) \geq $ A and  $P(A \cup B) \geq B$}   \\
	\text{The minimum value of }P(A \cup B)  \quad \text{will be} \quad MAX(0.7, 0.8)  = 0.8\\
	\text{largest possible value of}  P(A \cap B) = 1.5 - 0.8 = 0.7\\
	\text{Smallest possible value of } P(A \cap B) = 1.5 - 1 = 0.5\\              
	\boldsymbol{0.5 \quad \text{to} \quad 0.7}
	\end{align*}

	\section*{Problem 9} 
	9. Two events A and B are such that the probability of at least one of them occurring is 0.8, the probability of exactly one of them occurring is 0.6, and  $P(A\mid B) = 0.6$  Find the probability of A and the probability of B. 
	
	\begin{align*}
	P(A\cup B) &= 0.8 \\
	P(A \cup B \backslash A \cap B) &= .6 \rightarrow (hence \quad P(A \cap B) = 0.2)\\
	P(A \mid B) &= 0.6\\
	P(A \mid B) &= \frac{P(A \cap B)}{P(B)})\\
	0.6 &= \frac{0.2}{P(B)}\\
	P(B) &= \frac{0.2}{0.6} \\
	&= \boldsymbol{\frac{1}{3}}\\
	P(A \cup B) &= P(A) + P(B) - P(A \cap B) \\
	0.8 &= P(A) + \frac{1}{3} - 0.2 \\
	P(A) &= \boldsymbol{0.667}\\
	\end{align*}
	
	\section*{Problem 10} 

	10. (SOA) A survey of a group’s viewing habits over the last year revealed the following
information: 

%Roman numbers
\begin{enumerate}[label=(\roman*)]
		\item 28\% watched gymnastics
    \item 29\% watched baseball
    \item 19\% watched soccer
    \item 14\% watched gymnastics and baseball
    \item 12\% watched baseball and soccer
    \item 10\% watched gymnastics and soccer
    \item  8\% watched all three sports.
\end{enumerate}

Calculate the percentage of the group that watched none of the three sports during the last year.

\begin{align*}
g &= \text{Watched Gymnastics}\\
b &= \text{Watched Basebal}\\
s &= \text{watched Soccer}\\
P((g \cup b \cup s)^c) &= 1 - P(g \cup b \cup s)\\
&= 1-[P(g) + P(b) + P(S) - P(g \cap b) - P(g \cap s) - P(b \cap s) + P(g \cap b \cap s)] \\
&= 1- [.28 + .29 + .19 - .14 - .10 - .12 + .08] \\
&= \boldsymbol{0.52}
\end{align*}
\end{document}